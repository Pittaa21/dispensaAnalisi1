\documentclass{article}
\usepackage[most]{tcolorbox}
\usepackage{xcolor, array}
\usepackage[table]{xcolor}
\usepackage{graphicx, float}
\usepackage[T1]{fontenc}
\usepackage[utf8]{inputenc}
\usepackage[sfdefault]{atkinson}
\graphicspath{{images/}}
\usepackage{amsthm, amssymb, amsmath}
\usepackage[a4paper]{geometry}
\usepackage[dvipsnames]{xcolor}
\usepackage[skip=8pt, indent= 0pt]{parskip}
\usepackage{pgfplots}
\pgfplotsset{compat=1.18, width=10cm}
 \geometry{
 a4paper,
 total={170mm,257mm},
 left=20mm,
 top=20mm,
 }
\title{Analisi Matematica}
\author{Marco Pittarello}
\date{}

\newtcbtheorem{theo}{Teorema}{%
    colframe=BrickRed!100!white,
    colback=red!5!white
}{th}

\newtcbtheorem{ex}{Esempio}{%
    colframe=OliveGreen!100!white,
    colback=OliveGreen!5!white
}{ex}

\newtcbtheorem{defi}{Definizione}{%
    colframe=blue!100!white,
    colback=blue!10!white
}{def}
\renewcommand{\arraystretch}{2}
\setlength{\tabcolsep}{30pt}
\begin{document}
\maketitle
\tableofcontents
\newpage
\section{Principio di Induzione}

\begin{defi}{}{}
Il principio di induzione è un metodo per dimostrare predicati matematici.
\end{defi}
come \\
\begin{itemize}
    \item []$\forall n \in \mathbb{N} \qquad \underbrace{1 + 2 + 3 + \dots + n \ = \ \frac{n\ (n+1)}{2}}_{\text{P}(n)}$\\
    \item []$\forall n \in \mathbb{N} \qquad \underbrace{\forall x \in \mathbb{R} \quad x >  1 \quad (1+x)^n \ge nx+1}_{\text{P}(n)}$
\end{itemize}
\begin{theo}{1° forma}{}
Sia P$(n)$ un predicato con parametro $n \in \mathbb{N}$ e tale che:
\begin{enumerate}
    \item P$(0)$ è vero (\colorbox{yellow}{caso base})
    \item $\forall n \in \mathbb{N}\quad \text{P}(n) \xrightarrow{}\text{P}(n+1)$ (\colorbox{yellow}{passo induttivo})
\end{enumerate}
Allora $P(n)$ è vera $\forall n \in \mathbb{N}$
\end{theo}\ 

\begin{ex}{}{}
Dimostrare che $\forall n \in \mathbb{N} \quad \underbrace{2^n \ge n+1}_{\text{P}(n)}$.
\end{ex}
\underline{CASO BASE} : \quad P$(0) \quad 2^0 \ge 1 \quad \text{vero}$

\underline{PASSO INDUTTIVO} : \quad  $\forall n \in \mathbb{N} \quad \text{P}(n) \xrightarrow{} \text{P}(n+1)$\\

\begin{itemize}
    \item[] Suppongo che $2^n \ge n+1$ e dimostro che $2^{n+1} \ge n +2$
    \item[] $2^n \ge n+1\ \xrightarrow{}\ 2\cdot 2^n \ge 2 \cdot (n+1) $
    \item[] $2^{n+1}\ \ge\ 2n+2\ge n+2$
    \item[] Dunque abbiamo dimostrato che P$(n)\xrightarrow{}\text{P}(n+1)$
    \item[] Dunque per il principio di induzione è vero che $\forall n \in \mathbb{N}$ vale P$(n)$\\
\end{itemize}

\begin{ex}{}{}
    Dimostrare che $\forall n \in \mathbb{N}$ : \[\sum^n_{k=0}k = \frac{n(n+1)}{2}\]
\end{ex}
\underline{CASO BASE} : P$(0):\ $"$0$ = $0$" è vera

\underline{PASSO INDUTTIVO} :\quad Assumo che P$(n)$ è vera e dimostro che è vera anche P$(n+1)$

    \[\text{ovvero}\qquad \sum^{n+1}_{k=0}k=\frac{(n+1)(n+2)}{2}\]

    \[\sum^{n+1}_{k=0}k=\ \sum^{n}_{k=0}k+(n+1)\ =\ \frac{n(n+1)}{2}+(n+1)=\ \frac{n(n+1)+2(n+1)}{2}=\ \frac{(n+2)(n+1)}{2}\]
Ho dimostrato CASO BASE e PASSO INDUTTIVO, dunque per il principio di induzione segue che: 
\[\forall n\in\mathbb{N}\quad \text{P}(n)\]
\begin{theo}{2° forma}{}
    Sia P$(n)$, $n \in\mathbb{N}$\quad un predicato tale che:
    \begin{enumerate}
        \item P$(0)$ è vera (\colorbox{yellow}{caso base})
        \item $\forall n\in\mathbb{N}$,\quad $n \ge 1$ (\colorbox{yellow}{passo induttivo})
    \end{enumerate}
    Se $\forall m\in\mathbb{N}\quad 0 \le m \le n\quad \text{P}(n)$ è vera allora lo è anche P$(m)$ (\colorbox{yellow}{ipotesi induttiva})\\Allora $\forall n\in\mathbb{N}\quad \text{P}(n)\quad $ è vera
\end{theo}
\colorbox{yellow}{OSSERVAZIONE} è una forma "più forte" della 1° forma, poichè per dimostrare P$(n)$ si usa la condizione che P$(m)$ vale per tutti gli $m < n$\\\\
\colorbox{yellow}{OSSERVAZIONE} in entrambe le forme del principio di induzione possiamo sostituire $0$ con qualunque $n_0\in\mathbb{N}$. Ovvero, se per un predicato P$(m)$ dimostriamo:
\begin{itemize}
    \item il caso base per $n_o$
    \item il passo induttivo $\forall n\ge n_0$
\end{itemize}
Allora possiamo concludere che $\forall n\ge n_0 \text{P}(n)$ è vera

\begin{ex}{}{}
    Dimostriamo che $\forall n\in\mathbb{N}\quad n\ge2\quad \underbrace{n\text{ si può scrivere come prodotto di numeri primi}}_{\text{P}(n)}$
\end{ex}
\underline{CASO BASE} : P$(2)$ è banalmente vera: 2 è un numero primo


\underline{PASSO INDUTTIVO} : dimostriamo che $\forall n\in\mathbb{N} \quad n\ge3\quad (\forall\quad1\le m<n\quad\text{P}(m)) \xrightarrow{}\text{P}(n)$
\begin{itemize}
    \item[] Ovvero, assumendo che P$(m)$ vale\quad$\forall\quad1\le m<n$, ovvero si può scrivere come prodotto di primi, dimostriamo che anche $n$ si scrive come prodotto di primi ci sono due casi.
    \item[] Se $n$ è primo allore è chiaramente prodotto di primi
    \item[] Se $n$ non è primo allora è divisibile per un numero $m_1$ con $m_1 \ne n$ e $m_1\ne1$, 
    \item[] in particolare $2\le m_1<n$
    \item[] Dunque $\exists m_2\in\mathbb{N}$ tale che
    \item[] $n=m_1m_2$ con $m_1$,$m_2$ diversi da $m$ e da $1$ 
    \item[] Inoltre $2\le m_2<n$
    \item[] perchè $m_1<n$\qquad$m_1>1$
    \item[] Per l'ipotesi induttiva P$(m_1)$ e P$(m_2)$ sono vere.
\end{itemize}
\section{Coefficenti Binomiali}
\begin{defi}{}{}
    Definiamo C$_{n,k}$ = numero totale di modi possibile, e si chiama:
\begin{itemize}
    \item[] \underline{numero di combinazioni di $n$ elementi di classe $k$} 
\end{itemize}
\vspace{5pt}Spesso C$_{n,k}$ viene anche denotato con il simbolo $\binom{n}{k}$, chiamato
\begin{itemize}
    \item[] \underline{coefficiente binomiale $n$ su $k$}
\end{itemize}
\end{defi}
\colorbox{yellow}{Quanto vale $\binom{n}{k}$?}

\[\forall n\in\mathbb{N}\quad \forall k\in\mathbb{N}\quad k\le n\qquad\binom{n}{k} = C_{n,k} =\frac{n!}{k!\ (n-k)!}\]
\subsection{Proprietà di $\binom{n}{k}$}
    \begin{theo}{}{}
        $\forall n\in\mathbb{N}\quad\forall k\in\mathbb{N}\quad1\le k\le n$ si ha:\\
        \begin{itemize}
            \item [] $\binom{n+1}{k}=\ \binom{n}{k}+\binom{n}{k-1}$
        \end{itemize}
    \end{theo}
\colorbox{yellow}{OSSERVAZIONE:} Abbiamo un altro metodo per calcolare $\binom{n}{k}\forall n\in\mathbb{N}\quad\forall k\le n$\\
Utilizzando il teorema e $\binom{n}{0}=1$ possiamo calcolare ogni valore di $\binom{n}{k}\quad\forall n\in\mathbb{N}\quad\forall k\le n$, evitando di dover calcolare ogni volta i fattoriali.
\begin{ex}{Triangolo di Tartaglia}{}
    Si costruisce elencando per righe i coefficienti binomiali, la riga $n$ è composta da $\binom{n}{0},\binom{n}{1},\dots,\binom{n}{m}$
\end{ex}
\begin{gather}
    \binom{0}{0} = 1\\
    \binom{1}{0} = 1\qquad\binom{1}{1} = 1\\
    \binom{2}{0}=1\qquad\binom{2}{1}=\ \binom{1}{0}+\binom{1}{2}=2\qquad\binom{2}{2}=1\\
    \binom{3}{0}=1\qquad\qquad\binom{3}{1}=3\qquad\qquad\quad\qquad\binom{3}{2}=3\qquad\qquad\binom{3}{0}=1
\end{gather}
\begin{defi}{Binomio di Newton}{}
    \[\forall n\in\mathbb{N}\text{ vale }\forall p,q\in\mathbb{R} \qquad (p+q)^n=\sum_{k=0}^{n}\binom{n}{k}p^kq^{n-k}\]
\end{defi}
\section{Limiti di funzioni}
    \begin{defi}{Definizione di intorno e limite}{}
        Si dice intorno sferico di $r$ con $r\in\mathbb{R}$, un intervallo ]$r-\epsilon,r+\epsilon$[ con $\epsilon>0$ ; $\epsilon$ viene detta raggio dell'intorno
        \begin{itemize}
            \item Se $r=+\infty$, si dice intorno di $+\infty$ un intervallo del tipo \quad ]$M,+\infty$[ \quad con $M>0$ ($M\in\mathbb{R}$)
            \item Se $r=-\infty$, si dice intorno di $-\infty$ un intervallo del tipo \quad ]$-\infty,-M$[ \quad con $M>0$ ($M\in\mathbb{R}$)
        \end{itemize}\vspace{5pt}
        Dato $f:\text{dom}f\rightarrow\mathbb{R},\ \text{dom}f\subseteq \mathbb{R},\ x_0\in\mathbb{R}$ è punto di accumulazione a dom$f,\ l\in\mathbb{\bar{R}}$\\
        Si dice che $f$ ha limite $l$ per $x\rightarrow x_0$ e si scrive:
        \[\lim_{x\rightarrow x_0}f(x)=l\]
    \end{defi}
    Se $\forall\ V$ intorno di $l$\quad$\exists\ U$ intorno di $x_0$ tale che se $x\in U\cap$ dom$f$, $x\ne x_0$ allora $f(x)\in V$\\
    \colorbox{yellow}{OSSERVAZIONE}: 
    \begin{itemize}
        \item Se $l\in\mathbb{R}$, si dice che $f$ ha limite \underline{finito} in $x_0$
        \item Se $l=+\infty$ o $l=-\infty$, allora $f$ si dice \underline{divergente} per $x\rightarrow x_0$
        \item Se $l=0$, allora si dice che $f$ è \underline{infinitesima} in $x_0$
    \end{itemize}
    \begin{theo}{Unicità del limite}{}
        Se $x_0,l_1,l_2\in\mathbb{\bar{R}}$ e
        \[\lim_{x\rightarrow x_0}f(x)=l_1\qquad\text{e}\qquad\lim_{x\rightarrow x_0}f(x)=l_2\]
        allora $l_1=l_2$
    \end{theo}
    \colorbox{yellow}{Dim}: Suppongo per assurdo che $l_1\ne l_2$
    \begin{itemize}
        \item [] $P_1$ la proposizione di separazione di $\mathbb{R}$
        \item [] $\exists\ V_1$ intorno di $l_1$
        \item [] $\exists\ V_2$ intorno di $l_2$
        \item [] tale che $V_1\ \cap\ V_2 = \varnothing $
    \end{itemize}
    D'altra parte\\
    \begin{minipage}{0.7\textwidth}
        \[\lim_{x\rightarrow x_0}f(x)=l_1\Rightarrow \]
    \end{minipage}
    \begin{minipage}{0.3\textwidth}
        $\exists\ U_1$ intorno di $x_0$ tale che\\$x\in U_1\ \cap\ \text{dom}f, x\ne x_0$\\ allora $f(x)\in V_1$
    \end{minipage}\vspace{10pt}
        \begin{minipage}{0.7\textwidth}
        \[\lim_{x\rightarrow x_0}f(x)=l_2\Rightarrow \]
    \end{minipage}
    \begin{minipage}{0.3\textwidth}
        $\exists\ U_2$ intorno di $x_0$ tale che\\$x\in U_2\ \cap\ \text{dom}f, x\ne x_0$\\ allora $f(x)\in V_2$
    \end{minipage}
    Pongo
    \begin{itemize}
        \item [] $U=U_1\ \cap\ U_2$ intorno di $x_0$
        \item [] $\Rightarrow\exists\ \bar{x}\in\quad U\ \smallsetminus\{ x_0\},\ \bar{x}\in\text{dom}f$
        \item [] $\Rightarrow f(\bar{x})\in V_1\ \cap\ V_2 = \varnothing$\qquad \colorbox{yellow}{ASSURDO}\quad poichè $\bar{x}\in U_1$ e $\bar{x}\in U_2$
    \end{itemize}
    \begin{ex}{Verificare}{}
        \[\lim_{x\rightarrow2}2\ |x-2|\ \cos[\ \ln |x-2| + e^{\sin x}]=0\]
    \end{ex}
    $x_0=2,\quad l=0$
    \begin{align*}
        &f(x)= 2\ |x-2|\ \cos[\ \ln|x-2|+e^{\sin x}]\\
        &\text{dom}f=\mathbb{R}\smallsetminus\{2\}
    \end{align*}
    Devo verificare che 
    $\forall\epsilon>0\quad\exists\ \delta $ tale che se $\underbrace{|x-2|<\delta\ ,\ x\in\mathbb{R}\smallsetminus\{2\},\ x\ne2}_{0<|x-2|<\delta}$ 
    \quad allora \quad $\underbrace{|f(x)-0|<\epsilon}_{|f(x)<\epsilon|}$\\
    ossia \[|2\ |x-2|\ \cos[\ \ln|x-2|+e^{\sin x}]|<\epsilon\quad \Rightarrow\quad \circledast \]
    Parto da $\circledast$ \[2\ |x-2|\ \underbrace{|cos[\ \ln|x-2|+e^{\sin x}]|}_{\le\ 1}\quad\le\quad 2\ |x-2|\quad<\quad\epsilon\]
    Voglio $\delta>0$ tale che se $|x-2|<\delta$ e $x\ne2$\quad allora\quad $2\ |x-2|<\epsilon\quad\Leftrightarrow\quad|x-2|<\frac{\epsilon}{2}$

    Prendo $\delta=\frac{\epsilon}{2}$ e ho verificato che vale il limite.
    \begin{ex}{Limite destro e sinistro}{}
        $\sin x =$
        \begin{align*}
            -1\quad x<0\\
            0\quad x=0\\
            1\quad x>0
        \end{align*}
    \end{ex}
    \begin{minipage}{0.5\textwidth}
    \begin{tikzpicture}
        \begin{axis}[axis lines=middle, xmax=5,xmin=-5,ymax=5,ymin=-5,ylabel=$y$,xlabel=$x$]
            \addplot[color=blue, domain=0.3:100]{2};
            \addplot[color=blue, domain=-0.3:-100]{-2};
        \end{axis}
    \end{tikzpicture}
    \end{minipage}\hspace{50pt}
    \begin{minipage}{0.5\textwidth}
        Se $x\rightarrow0^-\quad \text{sgn}\ x\rightarrow-1$\\
        Se $x\rightarrow0^+\quad \text{sgn}\ x\rightarrow1$\\\\
        Quindi $\nexists \lim_{x\rightarrow0}\text{sgn}\ x$ perchè affinchè esista, i limiti destro e sinistro devono essere uguali
    \end{minipage}\vspace{20pt}
    \begin{defi}{Punto di Accumulazione}{}
        Sia $A\subseteq \mathbb{R},\ r\in\mathbb{R}$ si dice:
        \begin{itemize}
            \item \underline{punto di acc. destro per $A$} se $\forall\epsilon>0\quad A\ \cap\ ]r,r+\epsilon[\ \ne\varnothing$\quad(cioè $\exists\ a\in A$ tale che $r<a<r+\epsilon$)
            \item \underline{punto di acc. sinistro per $A$} se $\forall\epsilon>0\quad A\ \cap\ ]r-\epsilon,r[\ \ne\varnothing$\quad(cioè $\exists\ a\in A$ tale che $r-\epsilon<a<r$)
        \end{itemize}
    \end{defi}\vspace{20pt}
    \begin{ex}{Verificare che}{}
        \[\lim_{x\rightarrow0^+}e^{1/x}=+\infty\]
    \end{ex}
    \colorbox{yellow}{Soluzione}:\quad devo mostrare $\forall\ M>0\quad\exists\ \delta>0$ tale che se\quad$0<x<\underbrace{0+\delta}_{\delta}$\quad allora\quad$e^{1/x}>M$\\
    $e^{1/x}>M\Leftrightarrow\ln e^{1/x}>\ln M\Leftrightarrow\frac{1}{x}>\ln M\Leftrightarrow x<\underbrace{\frac{1}{\ln M}}_{\delta}$\\
    Prendo $\delta=\frac{1}{\ln M}$\\
    \begin{defi}{Limiti e valore assoluto}{}
        Sia $x_0\in\mathbb{\bar{R}},\quad l\in\mathbb{R},\quad x_0$ di acc. per dom$f$. Allora:\\
        \[\lim_{x\rightarrow x_0}f(x)=l\quad\Leftrightarrow\quad\lim_{x\rightarrow x_0}|f(x)-l|=0\]
    \end{defi}
    \colorbox{yellow}{Prop}: Sia $x_0,\ l\in\mathbb{\bar{R}}$. Se $\lim_{x\rightarrow x_0}f(x)=l$, allora:\\
    \[\lim_{x\rightarrow x_0}|f(x)|=|l|\]
    \colorbox{yellow}{Osservazione}: \[\lim_{x\rightarrow x_0}|f(x)|=|l|\quad\nRightarrow\quad\lim_{x\rightarrow x_0}f(x)=\pm l\]\\
    \begin{theo}{Permanenza del segno}{}
        Dato $f$ reale di variabile reale, $x_0\in\mathbb{\bar{R}}$, di acc. per dom$f$ e supp.
        \[\lim_{x\rightarrow x_0}f(x)=l\quad>\ 0 \]
        Allora $\exists\ U$ intorno di $x_0$ tale che se $x\in U\ \cap$ dom$f$, $x\ne x_0$, allora\[f(x)>0\]
    \end{theo}
    \colorbox{yellow}{Dim}:\quad Considero il caso $l\in\mathbb{R},\ x_0\in\mathbb{\bar{R}}$. Poichè $l>0\Rightarrow\ \exists\ V$ intorno di $l$ tale che:
    \[V\subseteq\ ]0,+\infty[\]
    $\exists\ U$ intorno di $x_0$ tale che se $x\in U\ \cap$ dom$f,\ x\ne x_0$ allora $f(x)\in V$\quad Poichè $V\subseteq\ ]0,+\infty[$, ho
    \[f(x>0)\]
    $\forall x\in U$ dom$f,\ x\ne x_0$\\
    \colorbox{yellow}{Oss}: vale l'analogo con $l<0$\\
    \begin{theo}{Teorema del Confronto}{}
        Siano $f,\ g$ funzioni reali di variabile reale, $x_0$ di acc. per dom$f\ \cap$ dom$g$, tale che
        \[f(x)\le g(x)\qquad\text{definitivamente per } x\rightarrow x_0\]
        Se \[l_f=\lim_{x\rightarrow x_0}f(x)\qquad\text{e}\qquad l_g=\lim_{x\rightarrow x_0}g(x)\]
        allora\[l_f\le l_g\]
    \end{theo}
    \colorbox{yellow}{Oss}: se $f(x)<g(x)$\quad definitivamente per \LARGE{$x\rightarrow x_0\brace \exists\ l_f\ \text{e}\ l_g$$\nRightarrow$}\large\quad $l_f<l_g$\\\normalsize
    \begin{ex}{}{}
        $0<e^x\qquad\forall x\in\mathbb{R}\qquad\lim_{x\rightarrow-\infty}e^x=0=\lim_{x\rightarrow-\infty}0$
    \end{ex}\vspace{10pt}
    \begin{theo}{Teorema dei due carabinieri}{}
        Sia $X\subseteq\mathbb{R},\ f,g,h\ =X\rightarrow\mathbb{R},\ x_0\in\mathbb{\bar{R}}$ di acc. a $X$. Se
        \large\[\circledast\qquad f(x)\le h(x)\le g(x)\qquad \text{def. per }x\rightarrow x_0\]\normalsize
        e\large\[\lim_{x\rightarrow x_0}f(x)=\lim_{x\rightarrow x_0}g(x)=l\]\normalsize
        allora\large\[\lim_{x\rightarrow x_0}h(x)=l\]\normalsize
    \end{theo}
    \colorbox{yellow}{Dim}: Suppongo per semplicità che $\circledast$ valga $\forall x\in X$. Devo mostrare che $\forall V$ intorno di $l\quad\exists\ U$ intorno di $x_0$ t.c.
    se $x\in(U\ \cap\ \underbrace{\text{dom}f}_{X})\smallsetminus\{x_0\}$, allora $h(x)\in V$ con $V$ intorno di $l$\\
    $\lim_{x\rightarrow x_0}f(x)=l \Rightarrow\ \exists\ U_f$ intorno di $x_0$ tale che se $x\in(U_f\ \cap\ X)\smallsetminus\{x_0\}$ allora $f(x)\in V$\\
    $\lim_{x\rightarrow x_0}g(x)=l \Rightarrow\ \exists\ U_g$ intorno di $x_0$ tale che se $x\in(U_g\ \cap\ X)\smallsetminus\{x_0\}$ allora $g(x)\in V$\\
    Prendo $U=U_f\ \cap\ U_g$ è intorno di $x_0$ se $x\in(U\ \cap\ X)\smallsetminus\{x_0\}\Rightarrow f(x)\in V,\quad g(x)\in V$\\
    Quindi $f(x)\le h(x)\le g(x)$ e $V$ intervallo\qquad allora $h(x)\in V$
    \begin{ex}{Teorema due carabinieri}{}
        \[\lim_{x\rightarrow0}\frac{\sin x}{x}=1\]
    \end{ex}
    \colorbox{yellow}{Svolgimento}:\qquad Si sfrutta il fatto che:\[\cos x\le\frac{\sin x}{x}\le1\qquad\text{def. per }x\rightarrow0\]
    \begin{ex}{Dimostrare il limite notevole:}{}
        \[\lim_{x\rightarrow0}\frac{1-\cos x}{x^2}=\frac{1}{2}\]
    \end{ex}
    \colorbox{yellow}{Svolgimento}: 
    \begin{align*}
        &\lim_{x\rightarrow0}\frac{1-\cos x}{x^2} = \lim_{x\rightarrow0}\frac{1-\cos x}{x^2}\cdot\frac{1+\cos x}{1+\cos x}=\\
        =&\lim_{x\rightarrow0}\frac{1-\cos^2x}{x^2}\cdot\frac{1}{1+\cos x} =\\
        =&\lim_{x\rightarrow0}\underbrace{\frac{\sin^2x}{x^2}}_{1}\cdot\underbrace{\frac{1}{1+\cos x}}_{1/2}= \frac{1}{2}
    \end{align*}
    \begin{defi}{Limite notevole}{}
        \[\lim_{x\rightarrow\pm\infty}(1+\frac{1}{x})^x=e\]
    \end{defi}
    \begin{theo}{Teorema della funzione composta o del cambio di variabile}{}
        Siano $f,g$ funzioni reali di variabile reale, tale che $g$ o $f$ sia definita in un insieme $X$ che abbia $x_0\in\mathbb{\bar{R}}$ come punto di acc. supp.\\
        \begin{enumerate}
            \item $\lim_{y\rightarrow y_0}g(y)=l$
            \item $\lim_{x\rightarrow x_0}f(x)=y_0$
            \item $f(x)\ne y_0$ def. per $x\rightarrow x_0$
        \end{enumerate}
        allora
        \begin{itemize}
            \item [] $\lim_{x\rightarrow x_0}g(f(x))=\lim_{y\rightarrow y_0}g(y)=l$\qquad con $y=f(x)$
        \end{itemize}
    \end{theo}
    \begin{ex}{}{}
        \[\lim_{x\rightarrow0}g(x\cdot\sin\frac{1}{x})\text{ non esiste}\]
    \end{ex}
    $g(y)=\qquad\cos y$ se $y\ne0$\qquad$0$ se $y=0$\newpage
    \begin{theo}{Operazioni sui limiti}{}
        $X\subseteq \mathbb{R},\ f,g$: $X\rightarrow\mathbb{R},\ x_0$ punto di acc. per $X,\ x_0\in\mathbb{\bar{R}}$. Supp.\\
        \[\lim_{x\rightarrow x_0}f(x)=l_f,\quad\lim_{x\rightarrow x_0}g(x)=l_g,\quad l_f,\ l_g\in\mathbb{R}\]
        Allora:
    \end{theo}
    \begin{enumerate}
        \item $c\in\mathbb{R},\quad\lim_{x\rightarrow x_0}(c\ f(x))=c\ l_f$
        \item $\lim_{x\rightarrow x_0}(f(x)\pm g(x))=l_f\pm l_g$
        \item $\lim_{x\rightarrow x_0}(f(x)\cdot g(x))=l_f\cdot l_g$
        \item $\lim_{x\rightarrow x_0}\frac{f(x)}{g(x)}=\frac{l_f}{l_g}\qquad$se $l_g\ne0$
        \item $\lim_{x\rightarrow x_0}\frac{1}{f(x)}=\frac{1}{l_f}$\qquad se $l_f\ne0$
    \end{enumerate}
    \colorbox{yellow}{Oss}: il teorema rimane vero se faccio limite destro o sinitro\\
    \begin{ex}{Calcolare}{}
        \[\lim_{x\rightarrow x_0}(f(x))^{g(x)}=(l_f)^{l_g}\]
        con \qquad$\lim_{x\rightarrow x_0}f(x)=l_f>0$\hfill$\lim_{x\rightarrow x_0}g(x)=l_g\in\mathbb{R}$
    \end{ex}
    \colorbox{yellow}{Svolgimento}:
    \large\begin{align*}
        &\lim_{x\rightarrow x_0}f(x)^{g(x)}=\lim_{x\rightarrow x_0}e^{\ln((f(x)^{g(x)})}=\\
        =&\lim_{x\rightarrow x_0}e^{g(x)\cdot\ln f(x)}=\lim_{t\rightarrow l_g\ln l_f}e^t=\\
        =&e^{l_g\ln l_f}=e^{\ln((l_f)^{l_g})}=\\
        =&l_f^{l_g}
    \end{align*}
    \normalsize
    \\\colorbox{yellow}{Proposizione}:\quad Sia $X\subseteq\mathbb{R},\ f,g:\ X\rightarrow\mathbb{R},\ x_0\in\mathbb{\bar{R}}$ 
    punto di acc. per $X,\ f$ infinitesima in $x_0$ (cioè $\lim_{x\rightarrow x_0}f(x)=0$) e $g$ sia limitata def. per $x\rightarrow x_0$.
    Allora: \[\lim_{x\rightarrow x_0}f(x)\cdot g(x)=0\]
    \large\begin{center}
        ("prodotto di $f$. infinitesima per $f$. limitata è infinitesimo")
    \end{center}\normalsize\vspace{10pt}
    \begin{ex}{}{}
        \[\lim_{x\rightarrow+\infty}2^x\cdot\sin\frac{1}{2^x+1}\]
    \end{ex}
    \begin{align*}
        &\lim_{x\rightarrow+\infty}[(2^x)\sin(\frac{1}{2^x+1})+\sin(\frac{1}{2^x+1})-\sin(\frac{1}{2^x+1})]=\lim_{x\rightarrow+\infty}[(2^x+1)\sin(\frac{1}{2^x+1})-\sin(\frac{1}{2^x+1})]=\\
        &y=\frac{1}{2^x+1}\qquad y\rightarrow0 \text{ per }x\rightarrow+\infty\\
        =&\lim_{y\rightarrow0}[\frac{1}{y}\sin y-\sin y]=\lim_{y\rightarrow0}[\underbrace{\frac{\sin y}{y}}_1-\sin y]=\\
        =&1-0=1
    \end{align*}
    \begin{ex}{Verificare che}{}
        \[\lim_{x\rightarrow0}\frac{\arctan x}{x}=1\]
    \end{ex}
    \colorbox{yellow}{Svolgimento}: Pongo $y=\arctan x\Leftrightarrow x=\tan y$\\
    Quindi\[\lim_{x\rightarrow0}\frac{\arctan x}{x}=\lim_{y\rightarrow0}\frac{y}{\tan y}=\lim_{y\rightarrow0}\frac{y}{\frac{\sin y}{\cos y}}=\]
    \[\lim_{y\rightarrow0}y\cdot\frac{\cos y}{\sin y}=\lim_{y\rightarrow0}\underbrace{(\frac{\sin y}{y})^{-1}}_{1}\cdot\cos y=1\cdot1=1\]
    \begin{ex}{Calcolare}{}
        \[\lim_{x\rightarrow0}\sqrt{|x|}\ \cos\frac{1}{x^2}\]
    \end{ex}
    \colorbox{yellow}{Soluzione}: il limite vale $0$ poichè $\sqrt{|x|}$ è infinitesima in $x=0$ e $\cos\frac{1}{x^2}$ è limitata def. per $x\rightarrow0$\\
    \begin{ex}{Calcolare}{}
        \[\lim_{x\rightarrow+\infty}(x+\sin x)\]
    \end{ex}
    \colorbox{yellow}{Soluzione}:\qquad $x+\sin x\ge \underbrace{x-1}_{+\infty}\qquad\forall x\in\mathbb{R}$\\
    Quindi anche: \[\lim_{x\rightarrow+\infty}x+\sin x=+\infty\]\\
    \begin{defi}{Forme Indeterminate}{}
        $f,g:\ X\rightarrow\mathbb{R},\quad x_0$ punto di acc. per $X$
        \[\lim_{x\rightarrow x_0}f(x)\cdot g(x)\]
        Non posso dire nulla se so che
        \[\lim_{x\rightarrow x_0}f(x)=0\qquad e\qquad\lim_{x\rightarrow x_0}g(x)=\pm\infty\qquad\text{F.I. }0\cdot\infty\]
        Oppure\[\text{F.I.}\qquad+\infty-\infty\qquad\frac{0}{0}\qquad\frac{\infty}{\infty}\qquad0^0\qquad+\infty^0\qquad1^{+\infty}\]
    \end{defi}
    \begin{ex}{Limite notevole}{}
        \[\lim_{x\rightarrow\pm\infty}(1+\frac{\alpha}{x})^x=e^x\qquad\forall\alpha\in\mathbb{R}\]
    \end{ex}
    \colorbox{yellow}{Soluzione}:\\
    \[\lim_{x\rightarrow\pm\infty}(1+\frac{\alpha}{x})^x=\lim_{y\rightarrow\pm\infty}(1+\frac{1}{y})^{\alpha y}=\lim_{y\rightarrow\pm\infty}(\underbrace{(1+\frac{1}{y})^y}_{e})^\alpha=e^\alpha\quad\checkmark\]
    \begin{ex}{Limite notevole}{}
        \[\lim_{x\rightarrow0}\frac{\ln(1+x)}{x}=1\]
    \end{ex}
    \colorbox{yellow}{Soluzione}:
    \[\lim_{x\rightarrow0}\frac{\ln(1+x)}{x}=\lim_{x\rightarrow0}\frac{1}{x}\ln(1+x)=\lim_{x\rightarrow0}\ln(\underbrace{(1+x)^{1/x}}_{e})=\ln e=1\qquad\checkmark\]\\
    \begin{ex}{Verificare il limite}{}
        \[\lim_{x\rightarrow0}\frac{e^x-1}{x}=1\]
    \end{ex}
    \colorbox{yellow}{Soluzione}:
    \begin{align*}
        &\lim_{x\rightarrow0}\frac{e^x-1}{x}=&y=e^x-1\\
        =&\lim_{y\rightarrow0}\frac{y}{\ln(1+y)}=&\text{uso il limite notevole verificato prima (esempio 18)}\\
        =&1\quad\checkmark
    \end{align*}
    \begin{defi}{Limite notevole}{}
        Dall'esempio precedente (es. 19) si ricava:\[\lim_{x\rightarrow0}\frac{\alpha^x-1}{x}=\ln\alpha\qquad\forall\alpha>0\]
    \end{defi}\vspace{10pt}
    \begin{ex}{}{}
        \[\lim_{x\rightarrow+\infty}(\frac{1+|\sin x|}{x})^x\qquad\text{F.I. }0^\infty\]
    \end{ex}
    \colorbox{yellow}{Svolgimento}: Uso il teorema del \underline{confronto}\\
    Visto che $0\le|\sin x|\le1$ allora $1\le 1+\sin x\le2$, quindi $\frac{1}{x}\le\frac{1+\sin x}{x}\le\frac{2}{x}$.\\
    I due estremi tendono a $0$ per $x\rightarrow+\infty$ quindi anche $\frac{1+|\sin x|}{x}$ tende a $0$.
    \\\begin{ex}{}{}
        \large\[\lim_{x\rightarrow-\infty}(1+3\sin e^x)^{\cos e^{-x^2}+\frac{1}{\sin e^x}}\]\normalsize
    \end{ex}
    \colorbox{yellow}{Svolgimento}:\\
    \LARGE\[\lim_{x\rightarrow-\infty}e^{(\cos e^{-x^2}+\frac{1}{\sin e^x})\ln(1+3\sin e^x)}\]\normalsize
    Ricordare il limite notevole $\frac{\ln(1+y)}{y}=1$ per $y\rightarrow0$Quindi:
    \[\lim_{x\rightarrow-\infty}[\cos e^{-x^2}\cdot\ln(1+3\sin e^x)+\frac{3}{3\sin e^x}\cdot\ln(1+3\sin e^x)]\]
    Adesso pongo $y=3\sin e^x$ che tende a $0$ per $x\rightarrow-\infty$ e ottengo:
    \[\cos e^{-x^2}=1\qquad\ln(1+3\sin e^x)=0\qquad3\cdot\underbrace{\frac{ln(1+y)}{y}}_1=3\]
    Il risultato è quindi $1\cdot0+3=3$\\
    \subsection{O-piccoli}
    \begin{defi}{O-Piccolo}{}
        Date $f,g$ funzioni reali di variabile reale, $x_0\in\mathbb{\bar{R}}$ di acc. per dom$f\ \cap$ dom$g$ , $g(x)\ne0$ def. per $x\rightarrow x_0$\\
        Si dice che $f$ è \underline{"o-piccolo" di $g$ per $x\rightarrow x_0$} se:
        \[\lim_{x\rightarrow x_0}\frac{f(x)}{g(x)}=0\]
        E si scrive:\quad$f(x)=o(g(x))$ per $x\rightarrow x_0$\quad$f(x)\in o(g(x))$ per $x\rightarrow x_0$
    \end{defi}
    \colorbox{yellow}{Osservazione}:\[\lim_{x\rightarrow x_0}f(x)=0\Leftrightarrow\lim_{x\rightarrow x_0}\frac{f(x)}{1}=0\Leftrightarrow f=o(1)\text{ per }x\rightarrow x_0\]\\
    \begin{ex}{o-piccolo}{}
        $1-\cos x=o(x)$ per $x\rightarrow0$?
    \end{ex}
    \colorbox{yellow}{Svolgimento}: \[\lim_{x\rightarrow0}\frac{1-\cos x}{x}=\lim_{x\rightarrow0}x\cdot\frac{1-\cos x}{x^2}\qquad x=0\qquad\frac{1-\cos x}{x^2}=\frac{1}{2}\]
    Quindi \qquad$0\cdot\frac{1}{2}=0\quad\checkmark$ e $1-\cos x$ è o-piccolo di $x$ per $x\rightarrow0$\\\\\\
    \colorbox{yellow}{Osservazione}:
    \begin{minipage}{0.4\textwidth}
        \begin{center}
            $f_1(x)=o[g(x)]$ per $x\rightarrow x_0$\\
            $f_2(x)=o[g(x)]$ per $x\rightarrow x_0$
        \end{center}
    \end{minipage}
    \begin{minipage}{0.1\textwidth}
        \large$\nRightarrow$
    \end{minipage}
    \begin{minipage}{0.4\textwidth}
        $f_1(x)=f_2(x)$ per $x\rightarrow x_0$
    \end{minipage}\\\\
    \colorbox{yellow}{Osservazione}:
    \[\lim_{x\rightarrow x_0}\frac{f(x)}{g(x)}=l\in\mathbb{R}\quad\Rightarrow\quad f(x)=lg(x)+o[g(x)]\quad\text{per }x\rightarrow x_0\]
    cioè \[f(x)-lg(x)=o[g(x)]\quad\text{per }x\rightarrow x_0\]\\
    \colorbox{yellow}{Dim}:\[\lim_{x\rightarrow x_0}\frac{f(x)-lg(x)}{g(x)}=\lim_{x\rightarrow x_0}(\underbrace{\frac{f(x)}{g(x)}}_l-\underbrace{\frac{lg(x)}{g(x)}}_l)=l-l=0\quad\checkmark\]\\
    \begin{theo}{Principio di sostituzione degli infinitesimi}{}
        $X\subseteq \mathbb{R}\ f,g,f_1,g_1:\ X\rightarrow\mathbb{R},\ x_0\in\mathbb{\bar{R}}$ punto di acc. per $X$.\\
        Se $g(x)\ne0$ def. per $x\rightarrow x_0$ e:\\
        \begin{itemize}
            \item [] $f(x)=f_1(x)+o(f_1(x))$\qquad per $x\rightarrow x_0$
            \item [] $g(x)=g_1(x)+o(g_1(x))$\qquad per $x\rightarrow x_0$
        \end{itemize}\vspace{8pt}
        Allora:\[\lim_{x\rightarrow x_0}\frac{f(x)}{g(x)}=\lim_{x\rightarrow x_0}\frac{f_1(x)}{g_1(x)}\]
    \end{theo}
    \colorbox{yellow}{Osservazione}:\[\lim_{x\rightarrow x_0}\frac{f(x)}{g(x)}=\lim_{x\rightarrow x_0}\frac{f_1(x)+o(f_1(x))}{g_1(x)+o(g_1(x))}=\lim_{x\rightarrow x_0}\frac{f_1(x)}{g_1(x)}\]\\
    \begin{ex}{Calcolare}{}
        \[\lim_{x\rightarrow0}\frac{\sin x-\ln(1+x)}{x\ln(1+x)-2\sin^2x+1-\cos x}\quad\text{F.I. }\frac{0}{0}\]
    \end{ex}
    \colorbox{yellow}{Svolgimento}: \\
    Conoscendo gli sviluppi in serie
    \begin{itemize}
        \item [] $\sin x=x-\frac{x^3}{6}+o(x^3)$ per $x\rightarrow0$
        \item [] $\ln (1+x)=x+o(x)$ per $x\rightarrow0$
    \end{itemize}
    NUM $=\sin x-\ln(1+x)=x-\frac{x^3}{6}+o(x^3)-x+\frac{x^2}{2}+o(x^2)=\frac{x^2}{2}-\frac{x^3}{6}+o(x^2)+o(x^3)=\frac{x^2}{2}+o(\frac{x^2}{2})$
    \begin{align*}
        &-\frac{x^3}{6}= o(\frac{x^2}{2})&o(x^3)=o(x^2)\\
        =\frac{x^2}{2}+o(\frac{x^2}{2 })
    \end{align*}
    DEN $=x\ln(1+x)-2\sin^2x+1-\cos x$\qquad $1-\cos x=\frac{1}{2}x^2+o(\frac{1}{2}x^2)$ per $x\rightarrow0$
    \begin{align*}
        &=x(x+o(x))-2(x+o(x))(x+o(x))+\frac{1}{2}x^2+o(\frac{x^2}{2})=&&&&&\\
        &=x^2+x\ o(x)-2x^2-4x\ o(x)-2(0(x))^2+\frac{1}{2}x^2+o(\frac{x^2}{2})=\\
        &=-\frac{1}{2}x^2+o(x^2)
    \end{align*}
    \[\lim_{x\rightarrow0}\frac{\text{NUM}}{\text{DEN}}=\lim_{x\rightarrow0}\frac{\frac{x^2}{2}+o(x^2)}{-\frac{1}{2}x^2+o(x^2)}=\lim_{x\rightarrow0}\frac{\frac{x^2}{2}}{-\frac{1}{2}x^2}=-1\]\\
    \newpage\begin{center}
        \LARGE\underline{Sviluppi asintotici di alcune funzioni}\\\vspace{10pt}\large
        {\rowcolors{1}{green!90!yellow!60}{green!80!yellow!50}
        \begin{tabular}{| m{5em} | m{20em} |}
            \hline
            $\sin x$ & $x-\frac{x^3}{3!}+\frac{x^5}{5!}+o(x^5)$ per $x\rightarrow0$\\\hline
            $\cos x$ & $1-\frac{x^2}{2!}+\frac{x^4}{4!}+o(x^4)$ per $x\rightarrow0$\\\hline
            $\ln(1+x)$ & $x-\frac{x^2}{2}+\frac{x^3}{3}+o(x^3)$ per $x\rightarrow0$\\\hline
            $\cosh x$ & $1-\frac{x^2}{2!}+\frac{x^4}{4!}+o(x^4)$ per $x\rightarrow0$\\\hline
            $\sinh x$ & $x-\frac{x^3}{3!}+\frac{x^5}{5!}+o(x^5)$ per $x\rightarrow0$\\\hline
            $\tan x$ & $x+\frac{x^3}{3}+o(x^3)$ per $x\rightarrow0$\\\hline
            $\arctan x$ & $x-\frac{x^3}{3}+$ per $x\rightarrow0$\\\hline
            $e^x$ & $1+x+\frac{x^2}{2!}+\frac{x^3}{3!}+o(x^3)$\\\hline
        \end{tabular}}
    \end{center}\vspace{10pt}
    \begin{ex}{Sviluppi asintotici}{}
        \[\sinh(e^x)\quad\text{per}\quad x\rightarrow-\infty\]
    \end{ex}
    \colorbox{yellow}{Svolgimento}:\quad Voglio sfruttare\[\sinh y=y+\frac{y^3}{3!}+\frac{y^5}{5!}+o(y^5)\]
    Quindi scrivo\[\sinh e^x=e^x+\frac{(e^x)^3}{3!}+\frac{(e^x)^5}{5!}+o((e^x)^5)\]
    Sapendo che $y$ deve tendere a $0$, abbiamo $e^x$ che tende a $0$ per $x\rightarrow-\infty$ quindi la sostituzione è valida\\
    \begin{defi}{Algebra degli o-piccoli}{}
        \begin{itemize}
            \item $o(g(x))+o(g(x))=o(g(x))$
            \item $o(g(x))-o(g(x))=o(g(x))$
            \item $o(g(x))\cdot o(g(x))=o(g(x)^2)$
            \item $g(x)\cdot o(g(x))=o(g(x)^2)$
        \end{itemize}
    \end{defi}
    \vspace{10pt}
    \large\underline{Confrontare esponenziali, logaritmi e potenze}\normalsize\\
    \[\lim_{x\rightarrow+\infty}\frac{e^x}{x^n}=+\infty\qquad\forall n\ge1\]
    $\Rightarrow x^n=o(e^x)$ per $x\rightarrow+\infty$
    \[\lim_{x\rightarrow+\infty}\frac{x^n}{\ln x}=+\infty\qquad\forall n\ge1\]
    $\Rightarrow\ln x=o(x^n)$ per $x\rightarrow+\infty$\\
    \underline{Quindi:}\[e^x>>x^n>>\ln x\qquad\text{per }x\rightarrow+\infty\]\\
    \begin{ex}{Calcolare}{}
        \[\lim_{x\rightarrow+\infty}\frac{x^5+e^x+\sin x}{3e^x+x^{15}\ln x}\qquad\text{F.I. }\frac{\infty}{\infty}\]
    \end{ex}
    \colorbox{yellow}{Svolgimento}: \\\\
    NUM:\qquad$x^5=o(e^x)\qquad\frac{x\sin x}{e^x}=\frac{x}{e^x}\cdot\sin x=0\cdot f$limitata $=0$\\
    $=e^x+o(e^x)$\\\\
    DEN:\qquad$\frac{x^{15}\ln x}{e^x}=\frac{x^{16}}{e^x}\cdot\frac{\ln x}{x}=0\cdot0=0$ quindi è $o(e^x)$\\
    $=3e^x+o(3e^x)$\\\\
    \[\lim_{x\rightarrow+\infty}\frac{e^x+o(e^x)}{3e^x+o(3e^x)}=\lim_{x\rightarrow+\infty}\frac{e^x}{3e^x}=\frac{1}{3}\]\\
    \begin{defi}{Nomenclatura}{}
        $f,g:\ X\rightarrow\mathbb{R},\ x_0\in\mathbb{\bar{R}}$ di acc. per $X$
    \end{defi}
    Se $\lim|f(x)|=+\infty,\ \lim|g(x)|=+\infty$\\\\
    \begin{minipage}{0.4\textwidth}
        \large\[\lim_{x\rightarrow x_0}|\frac{f(x)}{g(x)}|\]
    \end{minipage}
    \begin{minipage}{0.08\textwidth}
        \huge$\Rightarrow$
    \end{minipage}
    \begin{minipage}{0.5\textwidth}
        $+\infty$\qquad si dice che $f$ è un \underline{infinito di ordine superiore} a $g$ per $x\rightarrow x_0$\\\\
        $l\in\mathbb{R}\smallsetminus\{0\}$\qquad si dice che $f$ e $g$ sono \underline{infiniti dello stesso} \underline{ordine} per $x\rightarrow x_0$\\\\
        $0$\qquad si dice che $f$ è un \underline{infinito di ordine inferiore} a $g$ per $x\rightarrow x_0$\\\\
        $\nexists$\qquad si dice che $f$ e $g$ non sono \underline{confrontabili}
    \end{minipage}\\\\
    Se $\lim f(x)=0,\ \lim g(x)=0$\\\\
    \begin{minipage}{0.4\textwidth}
        \large\[\lim_{x\rightarrow x_0}|\frac{f(x)}{g(x)}|\]
    \end{minipage}
    \begin{minipage}{0.08\textwidth}
        \huge$\Rightarrow$
    \end{minipage}
    \begin{minipage}{0.5\textwidth}
        $+\infty$\qquad si dice che $f$ è un \underline{infinito di ordine inferiore} a $g$ per $x\rightarrow x_0$\\\\
        $l\in\mathbb{R}\smallsetminus\{0\}$\qquad si dice che $f$ e $g$ sono \underline{infiniti dello stesso} \underline{ordine} per $x\rightarrow x_0$\\\\
        $0$\qquad si dice che $f$ è un \underline{infinito di ordine superiore} a $g$ per $x\rightarrow x_0$\\\\
        $\nexists$\qquad si dice che $f$ e $g$ non sono \underline{confrontabili}
    \end{minipage}\\\\
    \begin{ex}{Verificare}{}
        $\sin x^2$ e $x^2$ sono infiniti dello stesso ordine per $x\rightarrow0$
    \end{ex}
    \colorbox{yellow}{Svolgimento}:
    \[\lim_{x\rightarrow0}\frac{\sin x^2}{x^2}=1\ \checkmark\qquad\text{Ricordando il limite notevole }\lim_{x\rightarrow0}\frac{\sin x}{x}=1\]\\
    \begin{ex}{Calcolare}{}
        \[\lim_{x\rightarrow0^+}\frac{4x^2\sin(\sqrt{x})+(1-\cos x)^2}{\sqrt{x}\ \sinh(x^2)+(e^x-1)^3}\]
    \end{ex}
    \colorbox{yellow}{Svolgimento}: Sapendo che
    \begin{itemize}
        \item [] $\sin y=y-\frac{y^3}{3!}+o(y^3)$\quad per $y\rightarrow0$
        \item [] $\cos y=1-\frac{y^2}{2!}+o(y^2)$\quad per $y\rightarrow0$
        \item [] $e^y=1+y+\frac{y^2}{2}+o(y^2)$\quad per $y\rightarrow0$
        \item [] $\sinh y=y+\frac{y^3}{3!}+o(y^3)$\quad per $y\rightarrow0$
    \end{itemize}
    \vspace{10pt}Allora NUM:\qquad$\sin(\sqrt{x})=x^{1/2}-\frac{x^{3/2}}{3!}+o(x^{3/2})$\qquad$(1-\cos x)^2=(1-1+\frac{x^2}{2}+o(x^2))^2$\qquad $x\rightarrow0$
    \begin{itemize}
        \item [] $=4x^{5/2}-\frac{4}{6}x^{7/2}+o(x^{7/2})+\frac{1}{4}x^4+o(x^4)=$
        \item [] $=4x^{5/2}+o(x^{5/2})$ per $x\rightarrow0^+$\qquad perchè devo prendere ciò che tende a $0$ più lentamente
    \end{itemize}
    \vspace{10pt}DEN:\qquad$\sqrt{x}\ \sinh(x^2)=x^{5/2}+\frac{1}{6}x^{9/2}+o(x^{9/2})\qquad(e^x-1)^3=(1-1+x+\frac{x^2}{2}+o(x^2))^3\qquad x\rightarrow0^+$
    \begin{itemize}
        \item [] $=x^{5/2}+o(x^{5/2})+x^3+o(x^3)=x^{5/2}+o(x^{5/2})$ per $x\rightarrow0^+$
    \end{itemize}
    \vspace{5pt}\[\Rightarrow\qquad\lim_{x\rightarrow0^+}\frac{4x^{5/2}+o(x^{5/2})}{x^{5/2}+o(x^{5/2})}=\lim_{x\rightarrow0^+}\frac{4}{1}=4\]
    \\\section{Successioni}
    \begin{defi}{Successione}{}
        Una successione è una funzione il cui dominio è $\mathbb{N}$ o un suo sottoinsieme infinito, per noi avranno valori in $\mathbb{R}$, ossia\[f:\ \mathbb{N}\rightarrow\mathbb{R}\]
        Le successioni si scrivono col simbolo $\{a_n\}_{n\in\mathbb{N}}$\\\\
        Si dice che: \[\lim_{n\rightarrow+\infty}a_n=l\]
        Se $\forall\ V$ intorno di $l\ \exists\ N>0$ tale che $a_n\in V\ \forall\ n>N$
    \end{defi}
    \colorbox{yellow}{Es}: $\{\frac{1}{n}\}_{n\ge1},\qquad1,\frac{1}{2},\frac{1}{3},\dots\rightarrow?\quad n\rightarrow+\infty$\\
    \\\colorbox{yellow}{Def}: sia $\{a_n\}_{n\in\mathbb{N}}$ una successione:
    \begin{enumerate}
        \item $\{a_n\}_{n\in\mathbb{N}}$ è \underline{convergente} se $\exists\ \lim_{}a_n\in\mathbb{R}$. Se $\lim_{}a_n=0$, la successione è \underline{infinitesima}
        \item $\{a_n\}_{n\in\mathbb{N}}$ è \underline{divergente} se $\lim_{}a_n=\pm\infty$
        \item $\{a_n\}_{n\in\mathbb{N}}$ è \underline{regolare} o \underline{determinata} se $\exists\ \lim_{}a_n$
        \item $\{a_n\}_{n\in\mathbb{N}}$ è \underline{irregolare} o \underline{indeterminata} se $\nexists\ \lim_{}a_n$
    \end{enumerate}
    \begin{ex}{}{}
        \[a_n=(1+\frac{1}{n})^n,\ n\ge1\]
    \end{ex}
    \colorbox{yellow}{Svolgimento}:\[\lim_{n\rightarrow+\infty}(1+\frac{1}{n})^n=e\]
    $a_n$ è \underline{convergente}.\\\\
    \colorbox{yellow}{Proposizione} sia $\{a_n\}_{n\in\mathbb{N}}$ succ. convergente, allora è anche limitata, ossia $\exists\ M>0$ tale che
    \[|a_n|\le M\qquad\forall\ n\in\mathbb{N}\]
    limitata $\nRightarrow$ convergente\\
    convergente $\Rightarrow$ limitata\\
    \begin{theo}{Teorema della permanenza del segno}{}
        Se $\lim a_n=l>0$, allora $a_n>0$ definitivamente per $n\rightarrow+\infty$
    \end{theo}
    \vspace{10pt}
    \begin{theo}{Teorema del Confronto}{}
        $\{a_n\},\{b_n\}$ siano due succ. tali che $a_n\le b_n$ def. per $x\rightarrow+\infty$ e
        \[\lim_{n\rightarrow+\infty}a_n=l_a\qquad\text{e}\qquad\lim_{n\rightarrow+\infty}b_n=l_b\]
        Allora \qquad$l_a\le l_b$
    \end{theo}
    \vspace{10pt}
    \begin{theo}{Teorema dei due carabinieri}{}
        $\{a_n\},\{b_n\},\{c_n\}$ succ. tali che $a_n\le b_n\le c_n$ def. per $n\rightarrow+\infty$ Se
        \[\lim_{n\rightarrow+\infty}a_n=\lim_{n\rightarrow+\infty}b_n=l\in\mathbb{\bar{R}}\]
        Allora\[\lim_{n\rightarrow+\infty}b_n=l\]
    \end{theo}
    \vspace{10pt}\colorbox{yellow}{Gerarchia degli infiniti}:
    \[n^n>>n!>>e^n>>n^k>>\log_bn\qquad\text{per }n\rightarrow+\infty\qquad a,b>1\qquad k>0\]
    \begin{defi}{Successioni monotone}{}
        $\{a_n\}_{n\in\mathbb{N}}$ succ. reale, si dice\\
        \begin{itemize}
            \item \underline{Crescente}: se $a_{n+1}\ge a_n\qquad\forall\ n\in\mathbb{N}$
            \item \underline{Decrescente}: se $a_{n+1}\le a_n\qquad\forall\ n\in\mathbb{N}$
            \item \underline{Strett. Crescente}: se $a_{n+1}> a_n\qquad\forall\ n\in\mathbb{N}$
            \item \underline{Strett. Decrescente}: se $a_{n+1}< a_n\qquad\forall\ n\in\mathbb{N}$
        \end{itemize}
    \end{defi}
    \newpage\colorbox{yellow}{Es}:
    \begin{itemize}
        \item [] $\{2^n\}=1,2,4,\dots$ \underline{Strett. crescente}
        \item [] $\{1^n\}$ \underline{Costante}
        \item [] $\{\frac{1}{2^n}\}$ \underline{Strett. decrescente}
        \item [] $\{1+\frac{1}{n}\}$ \underline{Strett. decrescente}
    \end{itemize}\vspace{10pt}
    \begin{defi}{Successione di Cauchy}{}
        Una succ. reale $\{a_n\}_{n\in\mathbb{N}}$ si dice di \underline{Cauchy} (o successione fondamentale) se $\forall\ \epsilon>0\quad\exists\ N>0$ tale che
        \[n,p>N\Rightarrow|a_n-a_p|<\epsilon\]
    \end{defi}
    \colorbox{yellow}{Es}: $a_n=(-1)^n$ non è di Cauchy\\\\
    $a_n=\frac{1}{1+n}$ è di Cauchy\\\\
    Una successione reale $a_n$ è convergente se e solo se è di Cauchy.\\
    \begin{defi}{Sottosuccessioni}{}
        Data una succ. $a_n$ si dice \underline{sottosuccessione} di $\{a_n\}_{n\in\mathbb{N}}$ 
        una succ. $\{a_{n_k}\}_{k\in\mathbb{N}}$ tale che $a_k$ sia una succ. strett. crescente di numeri naturali, cioè $n_{k+1}>n_k\qquad\forall\ k\in\mathbb{N}$
    \end{defi}
    \colorbox{yellow}{Es}: $n_k=2k,\ k\in\mathbb{N}\qquad a_{n_k}=a_{2k},\ k\in\mathbb{N}$\qquad prendo solo gli elementi pari\\\\
    \colorbox{yellow}{Osservazione}: per mostrare che una succ. non ha limite, mi basta mostrare che due sottosucc. hanno limite diverso
    o che una non abbia limite.\\
    \begin{theo}{Teorema di Bolzano-Weistrass}{}
        Sia $a_n$ una succ. limitata. Allora $\exists$ una sottosucc. di $a_n$ convergente
    \end{theo}
    \vspace{20pt}
    \begin{ex}{Calcolare}{}
        \[\lim_{n\rightarrow+\infty}\frac{\ln((n+5)!)-\ln(n!+5)}{\ln(n^\alpha+\cos(n\pi))}\]
        al variare di $\alpha>0$
    \end{ex}
    \vspace{10pt}\colorbox{yellow}{Svolgimento}: \\\\
    NUM = $\ln((n+5)!)-\ln(n!+5)=\quad\ln((n+5)(n+4)(n+3)(n+2)(n+1)n!)-\ln(n!(1+\frac{5}{n!})=$
    \begin{itemize}
        \item [] $=\ln((n+5)(n+4)(n+3)(n+2)(n+1)))+\ln(n!)-\ln(n!)-\underbrace{\ln(1+\frac{5}{n!})}_0=$
        \item [] $=\ln((n+5)(n+4)(n+3)(n+2)(n+1)))=$
        \item [] $=\ln(n^5)+o(\ln n^5)$
        \item [] $=5\ln n+o(\ln n)$\\
    \end{itemize}
    DEN = $\ln(n^\alpha+\underbrace{\cos(n\pi)}_{-1^n})=\quad\ln(n^\alpha(1+\frac{-1^n}{n^\alpha}))=$
    \begin{itemize}
        \item [] $=\ln n^\alpha\cdot\underbrace{\ln1+\frac{-1^n}{n^\alpha}}_0=$
        \item [] $=\ln n^\alpha=$
        \item [] $=\alpha\ln n+o(\ln n)$\\
    \end{itemize}
    \[\lim_{n\rightarrow+\infty}\frac{5\ln n+o(\ln n)}{\alpha\ln n+o(\ln n)}=\frac{5}{\alpha}\]
    \section{Funzioni Continue}
    \begin{defi}{Funzione Continua}{}
        $f$ funzione reale di variabile reale, $x_0\in\text{dom}f$. $f$ si dice \underline{continua in $x_0$}
        se è verificata una delle seguenti:\\
        \begin{itemize}
            \item $x_0$ è punto isolato del dominio
            \item Se $x_0$ non è punto isolato del dominio (ossia $x_0$ punto di acc. per dom$f$). Allora \[\lim_{x\rightarrow x_0}f(x)=f(x_0)\]
        \end{itemize}
        Una funzione è \underline{continua} se è continua in ogni punto del suo dominio.
    \end{defi}
    \vspace{10pt}\colorbox{yellow}{Osservazione}: Si parla di continuità di $f$ solo nei punti del dominio di $f$.\\
    \begin{ex}{}{}
        \[f(x)=\frac{1}{x}\qquad\text{dom}f=\mathbb{R}\smallsetminus\{0\}\]
    \end{ex}
    \[\lim_{x\rightarrow x_0}\frac{1}{x}=\frac{1}{x_0}=f(x_0)\quad\forall\ x_0\in\text{dom}f\]
    $\Rightarrow f$ è continua in $x_0\quad\forall\ x_0\in$ dom$f=\mathbb{R}\smallsetminus\{0\}$\\\\
    \colorbox{yellow}{Def}: Se $x_0\in$ dom$f$ è punto di acc. per dom$f$, ho che $f$ è continua in 
    $x_0\Leftrightarrow\forall\ \epsilon>0\ \exists\ \delta>0$ tale che se $|x-x_0|<\delta$ e $x\in$ dom$f$ allora:
    \[|f(x)-f(x_0)|<\epsilon\quad\Leftrightarrow\quad\lim_{x\rightarrow x_0}f(x)=f(x_0)\]\\
    \begin{defi}{Discontinuità di I specie}{}
        Sia $f$ funzione reale di variabile reale, $x_0\in$ dom$f$ punto di acc. destro e sinistro per dom$f$. Se esistono i limiti finiti:
        \[\lim_{x\rightarrow x_0^+}f(x)\quad e\quad\lim_{x\rightarrow x_0^-}f(x)\]
        Ma \[\lim_{x\rightarrow x_0^+}f(x)\quad \ne\quad\lim_{x\rightarrow x_0^-}f(x)\]
        Allora si dice che $x_0$ è punto di discontinuità di I specie o di salto
    \end{defi}
    \begin{defi}{Discontinuità di II specie}{}
        Sia $f$ funzione reale di variabile reale, $x_0\in$ dom$f$, punto di acc. destro e sinistro. Se:\\
        \begin{enumerate}
            \item Almeno uno dei due limiti sinistro o destro in $x_0$ sia infinito
            \item [] Oppure
            \item Almeno uno dei due limiti sinistro o destro in $x_0$ non esiste
        \end{enumerate}
        Allora si dice che $x_0$ è punto di discontinuità di II specie
    \end{defi}
    \vspace{20pt}
    \begin{defi}{Discontinuità di III specie}{}
        $I\subseteq \mathbb{R}$ intervallo, $f:I\rightarrow\mathbb{R}$, $x_0\in I$, $l\in\mathbb{R}$, tale che
        \[\lim_{x\rightarrow x_0}f(x)=l\ne f(x_0)\]
        Allora $x_0$ si dice punto di discontinuità eliminabile per $f$
    \end{defi}
    \vspace{20pt}
    \begin{theo}{Teorema di Weistrass}{}
        Sia $f$ funzione reale definita e continua su [$a,b$]. Allora $f$ ha massimo e minimo (assoluti) in [$a,b$],
        cioè $\exists\ x_m,x_M\in[a,b]$ tale che:
        \[\text{min}_{x\in[a,b]}f(x)=f(x_m)\qquad\text{max}_{x\in[a,b]}f(x)=f(x_M)\]
    \end{theo}
    \vspace{20pt}
    \begin{theo}{Teorema degli zeri}{}
        Sia $f:[a,b]\rightarrow\mathbb{R}$ continua in $[a,b]$ chiuso e limitato. Se $f(a)\cdot f(b)<0$\\
        Allora $\exists\ \epsilon\in]a,b[$ tale che $f(\epsilon)=0$
    \end{theo}
    \vspace{20pt}
    \begin{theo}{Teorema dei valori intermedi}{}
        Siano $I\subseteq\mathbb{R}$ intervallo $f:I\rightarrow\mathbb{R}$ continua. Allora $f(I)$ è un intervallo,
        cioè $\forall\ x_1,x_2\in I$ e $y\in\mathbb{R}$ tale che\[f(x_1)<y<f(x_2)\]
        Allora $\exists\ \epsilon\in I$ tale che $f(\epsilon)=y$
    \end{theo}
    \end{document}