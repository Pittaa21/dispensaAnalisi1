\documentclass{article}
\usepackage[most]{tcolorbox}
\usepackage{xcolor}
\usepackage{graphicx, float}
\usepackage[T1]{fontenc}
\usepackage[utf8]{inputenc}
\usepackage[sfdefault]{atkinson}
\graphicspath{{images/}}
\usepackage{amsthm, amssymb, amsmath}
\usepackage[a4paper]{geometry}
\usepackage[dvipsnames]{xcolor}
\usepackage[skip=8pt, indent= 0pt]{parskip}
\usepackage{pgfplots}
\pgfplotsset{compat=1.18, width=10cm}
 \geometry{
 a4paper,
 total={170mm,257mm},
 left=20mm,
 top=20mm,
 }
\title{Analisi Matematica}
\author{Marco Pittarello}
\date{}

\newtcbtheorem{theo}{Teorema}{%
    colframe=BrickRed!100!white,
    colback=red!5!white
}{th}

\newtcbtheorem{ex}{Esempio}{%
    colframe=OliveGreen!100!white,
    colback=OliveGreen!5!white
}{ex}

\newtcbtheorem{defi}{Definizione}{%
    colframe=blue!100!white,
    colback=blue!10!white
}{def}

\begin{document}
\maketitle
\tableofcontents
\newpage
\section{Principio di Induzione}

\begin{defi}{}{}
Il principio di induzione è un metodo per dimostrare predicati matematici.
\end{defi}
come \\
\begin{itemize}
    \item []$\forall n \in \mathbb{N} \qquad \underbrace{1 + 2 + 3 + \dots + n \ = \ \frac{n\ (n+1)}{2}}_{\text{P}(n)}$\\
    \item []$\forall n \in \mathbb{N} \qquad \underbrace{\forall x \in \mathbb{R} \quad x >  1 \quad (1+x)^n \ge nx+1}_{\text{P}(n)}$
\end{itemize}
\begin{theo}{1° forma}{}
Sia P$(n)$ un predicato con parametro $n \in \mathbb{N}$ e tale che:
\begin{enumerate}
    \item P$(0)$ è vero (\colorbox{yellow}{caso base})
    \item $\forall n \in \mathbb{N}\quad \text{P}(n) \xrightarrow{}\text{P}(n+1)$ (\colorbox{yellow}{passo induttivo})
\end{enumerate}
Allora $P(n)$ è vera $\forall n \in \mathbb{N}$
\end{theo}\ 

\begin{ex}{}{}
Dimostrare che $\forall n \in \mathbb{N} \quad \underbrace{2^n \ge n+1}_{\text{P}(n)}$.
\end{ex}
\underline{CASO BASE} : \quad P$(0) \quad 2^0 \ge 1 \quad \text{vero}$

\underline{PASSO INDUTTIVO} : \quad  $\forall n \in \mathbb{N} \quad \text{P}(n) \xrightarrow{} \text{P}(n+1)$\\

\begin{itemize}
    \item[] Suppongo che $2^n \ge n+1$ e dimostro che $2^{n+1} \ge n +2$
    \item[] $2^n \ge n+1\ \xrightarrow{}\ 2\cdot 2^n \ge 2 \cdot (n+1) $
    \item[] $2^{n+1}\ \ge\ 2n+2\ge n+2$
    \item[] Dunque abbiamo dimostrato che P$(n)\xrightarrow{}\text{P}(n+1)$
    \item[] Dunque per il principio di induzione è vero che $\forall n \in \mathbb{N}$ vale P$(n)$\\
\end{itemize}

\begin{ex}{}{}
    Dimostrare che $\forall n \in \mathbb{N}$ : \[\sum^n_{k=0}k = \frac{n(n+1)}{2}\]
\end{ex}
\underline{CASO BASE} : P$(0):\ $"$0$ = $0$" è vera

\underline{PASSO INDUTTIVO} :\quad Assumo che P$(n)$ è vera e dimostro che è vera anche P$(n+1)$

    \[\text{ovvero}\qquad \sum^{n+1}_{k=0}k=\frac{(n+1)(n+2)}{2}\]

    \[\sum^{n+1}_{k=0}k=\ \sum^{n}_{k=0}k+(n+1)\ =\ \frac{n(n+1)}{2}+(n+1)=\ \frac{n(n+1)+2(n+1)}{2}=\ \frac{(n+2)(n+1)}{2}\]
Ho dimostrato CASO BASE e PASSO INDUTTIVO, dunque per il principio di induzione segue che: 
\[\forall n\in\mathbb{N}\quad \text{P}(n)\]
\begin{theo}{2° forma}{}
    Sia P$(n)$, $n \in\mathbb{N}$\quad un predicato tale che:
    \begin{enumerate}
        \item P$(0)$ è vera (\colorbox{yellow}{caso base})
        \item $\forall n\in\mathbb{N}$,\quad $n \ge 1$ (\colorbox{yellow}{passo induttivo})
    \end{enumerate}
    Se $\forall m\in\mathbb{N}\quad 0 \le m \le n\quad \text{P}(n)$ è vera allora lo è anche P$(m)$ (\colorbox{yellow}{ipotesi induttiva})\\Allora $\forall n\in\mathbb{N}\quad \text{P}(n)\quad $ è vera
\end{theo}
\colorbox{yellow}{OSSERVAZIONE} è una forma "più forte" della 1° forma, poichè per dimostrare P$(n)$ si usa la condizione che P$(m)$ vale per tutti gli $m < n$\\\\
\colorbox{yellow}{OSSERVAZIONE} in entrambe le forme del principio di induzione possiamo sostituire $0$ con qualunque $n_0\in\mathbb{N}$. Ovvero, se per un predicato P$(m)$ dimostriamo:
\begin{itemize}
    \item il caso base per $n_o$
    \item il passo induttivo $\forall n\ge n_0$
\end{itemize}
Allora possiamo concludere che $\forall n\ge n_0 \text{P}(n)$ è vera

\begin{ex}{}{}
    Dimostriamo che $\forall n\in\mathbb{N}\quad n\ge2\quad \underbrace{n\text{ si può scrivere come prodotto di numeri primi}}_{\text{P}(n)}$
\end{ex}
\underline{CASO BASE} : P$(2)$ è banalmente vera: 2 è un numero primo


\underline{PASSO INDUTTIVO} : dimostriamo che $\forall n\in\mathbb{N} \quad n\ge3\quad (\forall\quad1\le m<n\quad\text{P}(m)) \xrightarrow{}\text{P}(n)$
\begin{itemize}
    \item[] Ovvero, assumendo che P$(m)$ vale\quad$\forall\quad1\le m<n$, ovvero si può scrivere come prodotto di primi, dimostriamo che anche $n$ si scrive come prodotto di primi ci sono due casi.
    \item[] Se $n$ è primo allore è chiaramente prodotto di primi
    \item[] Se $n$ non è primo allora è divisibile per un numero $m_1$ con $m_1 \ne n$ e $m_1\ne1$, 
    \item[] in particolare $2\le m_1<n$
    \item[] Dunque $\exists m_2\in\mathbb{N}$ tale che
    \item[] $n=m_1m_2$ con $m_1$,$m_2$ diversi da $m$ e da $1$ 
    \item[] Inoltre $2\le m_2<n$
    \item[] perchè $m_1<n$\qquad$m_1>1$
    \item[] Per l'ipotesi induttiva P$(m_1)$ e P$(m_2)$ sono vere.
\end{itemize}
\section{Coefficenti Binomiali}
\begin{defi}{}{}
    Definiamo C$_{n,k}$ = numero totale di modi possibile, e si chiama:
\begin{itemize}
    \item[] \underline{numero di combinazioni di $n$ elementi di classe $k$} 
\end{itemize}
\vspace{5pt}Spesso C$_{n,k}$ viene anche denotato con il simbolo $\binom{n}{k}$, chiamato
\begin{itemize}
    \item[] \underline{coefficiente binomiale $n$ su $k$}
\end{itemize}
\end{defi}
\colorbox{yellow}{Quanto vale $\binom{n}{k}$?}

\[\forall n\in\mathbb{N}\quad \forall k\in\mathbb{N}\quad k\le n\qquad\binom{n}{k} = C_{n,k} =\frac{n!}{k!\ (n-k)!}\]
\subsection{Proprietà di $\binom{n}{k}$}
    \begin{theo}{}{}
        $\forall n\in\mathbb{N}\quad\forall k\in\mathbb{N}\quad1\le k\le n$ si ha:\\
        \begin{itemize}
            \item [] $\binom{n+1}{k}=\ \binom{n}{k}+\binom{n}{k-1}$
        \end{itemize}
    \end{theo}
\colorbox{yellow}{OSSERVAZIONE:} Abbiamo un altro metodo per calcolare $\binom{n}{k}\forall n\in\mathbb{N}\quad\forall k\le n$\\
Utilizzando il teorema e $\binom{n}{0}=1$ possiamo calcolare ogni valore di $\binom{n}{k}\quad\forall n\in\mathbb{N}\quad\forall k\le n$, evitando di dover calcolare ogni volta i fattoriali.
\begin{ex}{Triangolo di Tartaglia}{}
    Si costruisce elencando per righe i coefficienti binomiali, la riga $n$ è composta da $\binom{n}{0},\binom{n}{1},\dots,\binom{n}{m}$
\end{ex}
\begin{gather}
    \binom{0}{0} = 1\\
    \binom{1}{0} = 1\qquad\binom{1}{1} = 1\\
    \binom{2}{0}=1\qquad\binom{2}{1}=\ \binom{1}{0}+\binom{1}{2}=2\qquad\binom{2}{2}=1\\
    \binom{3}{0}=1\qquad\qquad\binom{3}{1}=3\qquad\qquad\quad\qquad\binom{3}{2}=3\qquad\qquad\binom{3}{0}=1
\end{gather}
\begin{defi}{Binomio di Newton}{}
    \[\forall n\in\mathbb{N}\text{ vale }\forall p,q\in\mathbb{R} \qquad (p+q)^n=\sum_{k=0}^{n}\binom{n}{k}p^kq^{n-k}\]
\end{defi}
\section{Limiti di funzioni}
    \begin{defi}{Definizione di intorno e limite}{}
        Si dice intorno sferico di $r$ con $r\in\mathbb{R}$, un intervallo ]$r-\epsilon,r+\epsilon$[ con $\epsilon>0$ ; $\epsilon$ viene detta raggio dell'intorno
        \begin{itemize}
            \item Se $r=+\infty$, si dice intorno di $+\infty$ un intervallo del tipo \quad ]$M,+\infty$[ \quad con $M>0$ ($M\in\mathbb{R}$)
            \item Se $r=-\infty$, si dice intorno di $-\infty$ un intervallo del tipo \quad ]$-\infty,-M$[ \quad con $M>0$ ($M\in\mathbb{R}$)
        \end{itemize}\vspace{5pt}
        Dato $f:\text{dom}f\rightarrow\mathbb{R},\ \text{dom}f\subseteq \mathbb{R},\ x_0\in\mathbb{R}$ è punto di accumulazione a dom$f,\ l\in\mathbb{\bar{R}}$\\
        Si dice che $f$ ha limite $l$ per $x\rightarrow x_0$ e si scrive:
        \[\lim_{x\rightarrow x_0}f(x)=l\]
    \end{defi}
    Se $\forall\ V$ intorno di $l$\quad$\exists\ U$ intorno di $x_0$ tale che se $x\in U\cap$ dom$f$, $x\ne x_0$ allora $f(x)\in V$\\
    \colorbox{yellow}{OSSERVAZIONE}: 
    \begin{itemize}
        \item Se $l\in\mathbb{R}$, si dice che $f$ ha limite \underline{finito} in $x_0$
        \item Se $l=+\infty$ o $l=-\infty$, allora $f$ si dice \underline{divergente} per $x\rightarrow x_0$
        \item Se $l=0$, allora si dice che $f$ è \underline{infinitesima} in $x_0$
    \end{itemize}
    \begin{theo}{Unicità del limite}{}
        Se $x_0,l_1,l_2\in\mathbb{\bar{R}}$ e
        \[\lim_{x\rightarrow x_0}f(x)=l_1\qquad\text{e}\qquad\lim_{x\rightarrow x_0}f(x)=l_2\]
        allora $l_1=l_2$
    \end{theo}
    \colorbox{yellow}{Dim}: Suppongo per assurdo che $l_1\ne l_2$
    \begin{itemize}
        \item [] $P_1$ la proposizione di separazione di $\mathbb{R}$
        \item [] $\exists\ V_1$ intorno di $l_1$
        \item [] $\exists\ V_2$ intorno di $l_2$
        \item [] tale che $V_1\ \cap\ V_2 = \varnothing $
    \end{itemize}
    D'altra parte\\
    \begin{minipage}{0.7\textwidth}
        \[\lim_{x\rightarrow x_0}f(x)=l_1\Rightarrow \]
    \end{minipage}
    \begin{minipage}{0.3\textwidth}
        $\exists\ U_1$ intorno di $x_0$ tale che\\$x\in U_1\ \cap\ \text{dom}f, x\ne x_0$\\ allora $f(x)\in V_1$
    \end{minipage}\vspace{10pt}
        \begin{minipage}{0.7\textwidth}
        \[\lim_{x\rightarrow x_0}f(x)=l_2\Rightarrow \]
    \end{minipage}
    \begin{minipage}{0.3\textwidth}
        $\exists\ U_2$ intorno di $x_0$ tale che\\$x\in U_2\ \cap\ \text{dom}f, x\ne x_0$\\ allora $f(x)\in V_2$
    \end{minipage}
    Pongo
    \begin{itemize}
        \item [] $U=U_1\ \cap\ U_2$ intorno di $x_0$
        \item [] $\Rightarrow\exists\ \bar{x}\in\quad U\ \smallsetminus\{ x_0\},\ \bar{x}\in\text{dom}f$
        \item [] $\Rightarrow f(\bar{x})\in V_1\ \cap\ V_2 = \varnothing$\qquad \colorbox{yellow}{ASSURDO}\quad poichè $\bar{x}\in U_1$ e $\bar{x}\in U_2$
    \end{itemize}
    \begin{ex}{Verificare}{}
        \[\lim_{x\rightarrow2}2\ |x-2|\ \cos[\ \ln |x-2| + e^{\sin x}]=0\]
    \end{ex}
    $x_0=2,\quad l=0$
    \begin{align*}
        &f(x)= 2\ |x-2|\ \cos[\ \ln|x-2|+e^{\sin x}]\\
        &\text{dom}f=\mathbb{R}\smallsetminus\{2\}
    \end{align*}
    Devo verificare che 
    $\forall\epsilon>0\quad\exists\ \delta $ tale che se $\underbrace{|x-2|<\delta\ ,\ x\in\mathbb{R}\smallsetminus\{2\},\ x\ne2}_{0<|x-2|<\delta}$ 
    \quad allora \quad $\underbrace{|f(x)-0|<\epsilon}_{|f(x)<\epsilon|}$\\
    ossia \[|2\ |x-2|\ \cos[\ \ln|x-2|+e^{\sin x}]|<\epsilon\quad \Rightarrow\quad \circledast \]
    Parto da $\circledast$ \[2\ |x-2|\ \underbrace{|cos[\ \ln|x-2|+e^{\sin x}]|}_{\le\ 1}\quad\le\quad 2\ |x-2|\quad<\quad\epsilon\]
    Voglio $\delta>0$ tale che se $|x-2|<\delta$ e $x\ne2$\quad allora\quad $2\ |x-2|<\epsilon\quad\Leftrightarrow\quad|x-2|<\frac{\epsilon}{2}$

    Prendo $\delta=\frac{\epsilon}{2}$ e ho verificato che vale il limite.
    \begin{ex}{Limite destro e sinistro}{}
        $\sin x =$
        \begin{align*}
            -1\quad x<0\\
            0\quad x=0\\
            1\quad x>0
        \end{align*}
    \end{ex}
    \begin{minipage}{0.5\textwidth}
    \begin{tikzpicture}
        \begin{axis}[axis lines=middle, xmax=5,xmin=-5,ymax=5,ymin=-5,ylabel=$y$,xlabel=$x$]
            \addplot[color=blue, domain=0.3:100]{2};
            \addplot[color=blue, domain=-0.3:-100]{-2};
        \end{axis}
    \end{tikzpicture}
    \end{minipage}\hspace{50pt}
    \begin{minipage}{0.5\textwidth}
        Se $x\rightarrow0^-\quad \text{sgn}\ x\rightarrow-1$\\
        Se $x\rightarrow0^+\quad \text{sgn}\ x\rightarrow1$\\\\
        Quindi $\nexists \lim_{x\rightarrow0}\text{sgn}\ x$ perchè affinchè esista, i limiti destro e sinistro devono essere uguali
    \end{minipage}\vspace{20pt}
    \begin{defi}{Punto di Accumulazione}{}
        Sia $A\subseteq \mathbb{R},\ r\in\mathbb{R}$ si dice:
        \begin{itemize}
            \item \underline{punto di acc. destro per $A$} se $\forall\epsilon>0\quad A\ \cap\ ]r,r+\epsilon[\ \ne\varnothing$\quad(cioè $\exists\ a\in A$ tale che $r<a<r+\epsilon$)
            \item \underline{punto di acc. sinistro per $A$} se $\forall\epsilon>0\quad A\ \cap\ ]r-\epsilon,r[\ \ne\varnothing$\quad(cioè $\exists\ a\in A$ tale che $r-\epsilon<a<r$)
        \end{itemize}
    \end{defi}\vspace{20pt}
    \begin{ex}{Verificare che}{}
        \[\lim_{x\rightarrow0^+}e^{1/x}=+\infty\]
    \end{ex}
    \colorbox{yellow}{Soluzione}:\quad devo mostrare $\forall\ M>0\quad\exists\ \delta>0$ tale che se\quad$0<x<\underbrace{0+\delta}_{\delta}$\quad allora\quad$e^{1/x}>M$\\
    $e^{1/x}>M\Leftrightarrow\ln e^{1/x}>\ln M\Leftrightarrow\frac{1}{x}>\ln M\Leftrightarrow x<\underbrace{\frac{1}{\ln M}}_{\delta}$\\
    Prendo $\delta=\frac{1}{\ln M}$\\
    \begin{defi}{Limiti e valore assoluto}{}
        Sia $x_0\in\mathbb{\bar{R}},\quad l\in\mathbb{R},\quad x_0$ di acc. per dom$f$. Allora:\\
        \[\lim_{x\rightarrow x_0}f(x)=l\quad\Leftrightarrow\quad\lim_{x\rightarrow x_0}|f(x)-l|=0\]
    \end{defi}
    \colorbox{yellow}{Prop}: Sia $x_0,\ l\in\mathbb{\bar{R}}$. Se $\lim_{x\rightarrow x_0}f(x)=l$, allora:\\
    \[\lim_{x\rightarrow x_0}|f(x)|=|l|\]
    \colorbox{yellow}{Osservazione}: \[\lim_{x\rightarrow x_0}|f(x)|=|l|\quad\nRightarrow\quad\lim_{x\rightarrow x_0}f(x)=\pm l\]\\
    \begin{theo}{Permanenza del segno}{}
        Dato $f$ reale di variabile reale, $x_0\in\mathbb{\bar{R}}$, di acc. per dom$f$ e supp.
        \[\lim_{x\rightarrow x_0}f(x)=l\quad>\ 0 \]
        Allora $\exists\ U$ intorno di $x_0$ tale che se $x\in U\ \cap$ dom$f$, $x\ne x_0$, allora\[f(x)>0\]
    \end{theo}
    \colorbox{yellow}{Dim}:\quad Considero il caso $l\in\mathbb{R},\ x_0\in\mathbb{\bar{R}}$. Poichè $l>0\Rightarrow\ \exists\ V$ intorno di $l$ tale che:
    \[V\subseteq\ ]0,+\infty[\]
    $\exists\ U$ intorno di $x_0$ tale che se $x\in U\ \cap$ dom$f,\ x\ne x_0$ allora $f(x)\in V$\quad Poichè $V\subseteq\ ]0,+\infty[$, ho
    \[f(x>0)\]
    $\forall x\in U$ dom$f,\ x\ne x_0$\\
    \colorbox{yellow}{Oss}: vale l'analogo con $l<0$\\
    \begin{theo}{Teorema del Confronto}{}
        Siano $f,\ g$ funzioni reali di variabile reale, $x_0$ di acc. per dom$f\ \cap$ dom$g$, tale che
        \[f(x)\le g(x)\qquad\text{definitivamente per } x\rightarrow x_0\]
        Se \[l_f=\lim_{x\rightarrow x_0}f(x)\qquad\text{e}\qquad l_g=\lim_{x\rightarrow x_0}g(x)\]
        allora\[l_f\le l_g\]
    \end{theo}
    \colorbox{yellow}{Oss}: se $f(x)<g(x)$\quad definitivamente per \LARGE{$x\rightarrow x_0\brace \exists\ l_f\ \text{e}\ l_g$$\nRightarrow$}\large\quad $l_f<l_g$\\\normalsize
    \begin{ex}{}{}
        $0<e^x\qquad\forall x\in\mathbb{R}\qquad\lim_{x\rightarrow-\infty}e^x=0=\lim_{x\rightarrow-\infty}0$
    \end{ex}\vspace{10pt}
    \begin{theo}{Teorema dei due carabinieri}{}
        Sia $X\subseteq\mathbb{R},\ f,g,h\ =X\rightarrow\mathbb{R},\ x_0\in\mathbb{\bar{R}}$ di acc. a $X$. Se
        \large\[\circledast\qquad f(x)\le h(x)\le g(x)\qquad \text{def. per }x\rightarrow x_0\]\normalsize
        e\large\[\lim_{x\rightarrow x_0}f(x)=\lim_{x\rightarrow x_0}g(x)=l\]\normalsize
        allora\large\[\lim_{x\rightarrow x_0}h(x)=l\]\normalsize
    \end{theo}
    \colorbox{yellow}{Dim}: Suppongo per semplicità che $\circledast$ valga $\forall x\in X$. Devo mostrare che $\forall V$ intorno di $l\quad\exists\ U$ intorno di $x_0$ t.c.
    se $x\in(U\ \cap\ \underbrace{\text{dom}f}_{X})\smallsetminus\{x_0\}$, allora $h(x)\in V$ con $V$ intorno di $l$\\
    $\lim_{x\rightarrow x_0}f(x)=l \Rightarrow\ \exists\ U_f$ intorno di $x_0$ tale che se $x\in(U_f\ \cap\ X)\smallsetminus\{x_0\}$ allora $f(x)\in V$\\
    $\lim_{x\rightarrow x_0}g(x)=l \Rightarrow\ \exists\ U_g$ intorno di $x_0$ tale che se $x\in(U_g\ \cap\ X)\smallsetminus\{x_0\}$ allora $g(x)\in V$\\
    Prendo $U=U_f\ \cap\ U_g$ è intorno di $x_0$ se $x\in(U\ \cap\ X)\smallsetminus\{x_0\}\Rightarrow f(x)\in V,\quad g(x)\in V$\\
    Quindi $f(x)\le h(x)\le g(x)$ e $V$ intervallo\qquad allora $h(x)\in V$
    \begin{ex}{Teorema due carabinieri}{}
        \[\lim_{x\rightarrow0}\frac{\sin x}{x}=1\]
    \end{ex}
    \colorbox{yellow}{Svolgimento}:\qquad Si sfrutta il fatto che:\[\cos x\le\frac{\sin x}{x}\le1\qquad\text{def. per }x\rightarrow0\]
    \begin{ex}{Dimostrare}{}
        \[\lim_{x\rightarrow0}\frac{1-\cos x}{x^2}=\frac{1}{2}\]
    \end{ex}
    \colorbox{yellow}{Svolgimento}: 
    \begin{align*}
        &\lim_{x\rightarrow0}\frac{1-\cos x}{x^2} = \lim_{x\rightarrow0}\frac{1-\cos x}{x^2}\cdot\frac{1+\cos x}{1+\cos x}=\\
        &=\lim_{x\rightarrow0}\frac{1-\cos^2x}{x^2}\cdot\frac{1}{1+\cos x} =\\
        &=\lim_{x\rightarrow0}\underbrace{\frac{\sin^2x}{x^2}}_{1}\cdot\underbrace{\frac{1}{1+\cos x}}_{1/2}= \frac{1}{2}
    \end{align*}
\end{document}